\section{PLANTEAMIENTO DEL PROBLEMA }
\subsection{Antecedentes del problema}

En el contexto de la expansión de Alicorp hacia Centroamérica, específicamente en El Salvador, la empresa enfrenta desafíos significativos en la optimización de su arquitectura de aplicación y TI, especialmente en el área de logística. La plataforma móvil de Insuma, que facilita la gestión de logística y ventas B2B, debe integrar eficientemente con el sistema SAP para proporcionar información actualizada sobre inventarios y pedidos. Sin embargo, la alta latencia en la comunicación entre SAP y el frontend móvil, combinada con las limitaciones en la infraestructura de red, afecta la actualización y visualización de datos críticos. Estos problemas se agravan durante los picos de demanda, impactando negativamente la eficiencia operativa y la experiencia del usuario en el área logística.

En relación con la arquitectura de aplicación de la plataforma móvil de Insuma para el área de logística de Alicorp en El Salvador, se ha implementado una solución basada en tecnologías modernas para conectar el frontend móvil con el backend que interactúa con el sistema SAP. Los módulos de SAP necesarios incluyen SAP MM (Material Management) para la gestión de inventarios, SAP SD (Sales and Distribution) para la gestión de ventas y pedidos, y SAP WM (Warehouse Management) para la administración de almacenes. La arquitectura de software del frontend móvil se basa en React Native, que permite el desarrollo de una aplicación móvil nativa con una sola base de código. El backend, construido con Node.js y Express, actúa como intermediario entre el frontend y SAP, utilizando SAP Gateway (versión 7.50) para facilitar la integración con SAP HANA como base de datos.


Para la arquitectura de datos, se ha centrado en la implementación de SAP HANA como la base de datos principal, conocida por su capacidad para manejar grandes volúmenes de datos con alta velocidad de respuesta gracias a su almacenamiento en memoria. SAP HANA permite un procesamiento de datos en tiempo real y análisis de grandes conjuntos de datos con latencias muy bajas, lo cual es crucial para las operaciones logísticas de Alicorp. Para optimizar aún más la comunicación y reducir la latencia entre el backend y SAP, se ha introducido una capa de caching con Redis. Redis, que maneja datos en memoria y ofrece tiempos de respuesta extremadamente rápidos, almacena información frecuentemente solicitada, como inventarios y precios de productos, para evitar consultas repetidas a SAP HANA. Con Redis, la capacidad de gestionar datos en caché permite una recuperación rápida de información, mientras que SAP HANA gestiona grandes volúmenes de datos con una alta eficiencia. La aplicación móvil utiliza Axios, una biblioteca de JavaScript para solicitudes HTTP, para interactuar con el backend, accediendo a los datos almacenados en Redis para respuestas rápidas o directamente desde SAP HANA cuando se necesita la información más actualizada. 

Para la arquitectura de TI, el área de logística también emplea Google Cloud Platform (GCP) para mejorar la escalabilidad y la disponibilidad de sus servicios. Utiliza Google Compute Engine para la ejecución de instancias virtuales, específicamente máquinas virtuales N2 estándar con Ubuntu 20.04 LTS como sistema operativo. Estas instancias albergan tanto el backend de Node.js como la capa de caching de Redis. Google Cloud Storage se utiliza para el almacenamiento de datos no estructurados y archivos estáticos, mientras que Google Cloud SQL ofrece una base de datos relacional adicional con MySQL 8.0 para manejar datos complementarios. La arquitectura de GCP se basa en una distribución en múltiples zonas de disponibilidad (zonas de us-central1, us-east1 y us-west1) para asegurar la alta disponibilidad y la tolerancia a fallos.

El area de logística comparte una red VPN privada basada en OpenVPN para conectar sus instalaciones logísticas en El Salvador con los servidores centrales en Sudamérica. Esta red emplea routers Cisco 2901 y tiene un ancho de banda de 20 Mbps. Aunque la latencia promedio es de 250 ms, puede aumentar durante picos de uso, afectando la sincronización de datos en tiempo real. Durante las horas pico, la red en El Salvador puede experimentar algunas interrupciones, lo que afecta la capacidad de actualizar y visualizar datos en tiempo real, incrementando moderadamente los errores en pedidos y tiempos de resolución de problemas.

En el día a día de la operación de logística en Alicorp, los usuarios de la plataforma móvil de Insuma a menudo enfrentan problemas relacionados con la alta latencia en la comunicación entre el backend y SAP. Durante los picos de demanda, como en campañas de promociones especiales o eventos de ventas, la latencia puede aumentar considerablemente, lo que resulta en tiempos de respuesta más largos al acceder a información crítica, como la disponibilidad de productos o el estado de los pedidos. Por ejemplo, si un usuario intenta consultar el inventario en tiempo real para procesar un pedido urgente y la latencia es alta, puede experimentar retrasos significativos, mostrando datos desactualizados o incorrectos sobre la disponibilidad de productos. Esto no solo afecta la eficiencia operativa, sino que también impacta negativamente en la experiencia del cliente, que puede enfrentar largos tiempos de espera o errores en sus pedidos. Además, la carga de productos en la plataforma móvil, que depende de las actualizaciones provenientes de SAP, puede verse afectada por la sincronización lenta. Esto genera problemas adicionales, como la sobrecarga en el middleware o en los servidores de caching, exacerbando la latencia y retrasando la disponibilidad de los datos en la aplicación móvil, lo cual contribuye a una experiencia de usuario frustrante y a una menor eficiencia en el procesamiento de pedidos y gestión de inventarios.

\subsection{Definición del problema}
El área de logística de Alicorp enfrenta desafíos en la gestión de inventarios y pedidos debido a limitaciones en su arquitectura de aplicaciones y en su infraestructura de TI, afectando la integración de sistemas y la sincronización de datos. Estas deficiencias impactan negativamente los pilares de aplicación y TI, comprometiendo la eficiencia operativa.