\section{Diagnóstico de }
\subsection{Antecedentes del problema}

Alicorp, una empresa líder en la producción y comercialización de productos de consumo masivo en América Latina, ha decidido expandir sus operaciones hacia El Salvador como parte de su estrategia para aumentar su presencia en mercados clave de Centroamérica. Esta expansión enfrenta desafíos significativos relacionados con la arquitectura de TI y redes, que actualmente requieren ajustes para adaptarse a las demandas del nuevo mercado.

La migración a SAP S/4HANA ha sido un paso crucial para mejorar la eficiencia y agilidad en los procesos de Alicorp. SAP HANA, como base de datos en memoria, permite velocidades de procesamiento más rápidas en comparación con bases de datos tradicionales. Sin embargo, la integración con el módulo de gestión de la cadena de suministro (SCM) enfrenta limitaciones debido al uso de APIs que podrían estar obsoletas, lo que puede ralentizar la sincronización de datos. Aunque la sincronización de datos mediante procesos batch es adecuada para muchos escenarios, la frecuencia de actualización puede no ser suficiente para ciertos procesos críticos.

La empresa de logística adquirida en El Salvador opera con una versión anterior de SAP ERP (ECC 6.0), lo que puede generar desafíos en la integración con SAP S/4HANA. Esta versión antigua del sistema presenta ciertas dificultades para la transferencia eficiente de datos, aunque las soluciones de integración y migración pueden mitigar estos problemas. La infraestructura TI de esta empresa utiliza servidores con Windows Server 2012 R2, que puede limitar la flexibilidad y el rendimiento necesarios para operaciones logísticas optimizadas.

Un escenario típico podría implicar que un pedido registrado en SAP S/4HANA se procese con un cierto retraso debido a la sincronización con el SCM, aunque las demoras podrían no ser extremas. La empresa de logística en El Salvador enfrenta dificultades en la actualización de su sistema SAP ERP, lo que puede causar algunas inconsistencias en la planificación de rutas y la gestión de pedidos. Esto se traduce en un incremento moderado en la cantidad de solicitudes de actualización de estado de pedidos con inconsistencias y un aumento en las quejas de clientes por entregas retrasadas.

La infraestructura de red en El Salvador también presenta desafíos. Alicorp utiliza una red VPN privada basada en OpenVPN para conectar las instalaciones logísticas en El Salvador con los servidores centrales en Sudamérica. Esta red utiliza routers Cisco 2901 y tiene un ancho de banda de 20 Mbps. Aunque la latencia promedio es de 250 ms, puede aumentar durante picos de uso, afectando la sincronización de datos en tiempo real. Durante las horas pico, la red en El Salvador puede experimentar algunas interrupciones, lo que afecta la capacidad de actualizar y visualizar datos en tiempo real, incrementando moderadamente los errores en pedidos y tiempos de resolución de problemas.

En el día a día, Alicorp enfrenta problemas específicos debido a la infraestructura de red en El Salvador. Por ejemplo, durante una jornada laboral normal, un pedido urgente registrado en el sistema a las 10:00 AM podría no ser procesado correctamente hasta las 2:00 AM del día siguiente debido a la alta latencia y a la sincronización ineficiente de datos. Durante las horas pico, como entre las 9:00 AM y 11:00 AM, la red puede experimentar interrupciones y un aumento en la latencia a 500 ms, lo que impide la actualización en tiempo real del estado de los pedidos. Esto lleva a errores en el registro de pedidos, con un aumento en las solicitudes de actualización con inconsistencias y retrasos en la resolución de problemas logísticos, afectando la capacidad de cumplir con las expectativas de los clientes y deteriorando el servicio general.

\subsection{Definición del problema}

La expansión de Alicorp hacia El Salvador enfrenta desafíos en logística debido a limitaciones en la arquitectura de TI y en la infraestructura de red. La obsolescencia parcial del módulo SCM y la sincronización de datos en batch pueden causar ciertos retrasos en la gestión de pedidos, afectando la eficiencia operativa y la satisfacción del cliente. La infraestructura de red, con routers relativamente antiguos y ancho de banda limitado, presenta problemas de latencia y algunas interrupciones, que agravan los desafíos en la sincronización de datos y la resolución de problemas logísticos. Para garantizar una expansión exitosa y una operación eficiente en El Salvador, Alicorp debe abordar estos desafíos mediante la actualización de su infraestructura de TI y la mejora de su red, permitiendo una integración más fluida y una gestión logística más efectiva.