\section{Panteamiento del problema}
\subsection{Antecedentes del problema}

Alicorp, una empresa líder en la producción y comercialización de productos de consumo masivo en América Latina, ha decidido expandir sus operaciones hacia El Salvador como parte de su estrategia para aumentar su presencia en mercados clave de Centroamérica. Sin embargo, esta expansión enfrenta desafíos significativos debido a las limitaciones actuales en las arquitecturas de TI y redes, que no están completamente preparadas para manejar las demandas de un nuevo mercado. Estos problemas se han visto acentuados por la reciente transformación digital de la empresa, que incluyó la migración a SAP S/4HANA bajo el entorno de SAP HANA Enterprise Cloud, y la necesidad de coordinar múltiples integraciones con sistemas legacy.

En cuanto a la arquitectura de TI, Alicorp ha migrado a SAP S/4HANA para mejorar la eficiencia y agilidad de sus procesos. Sin embargo, el módulo de gestión de la cadena de suministro (SCM), desarrollado internamente y no actualizado desde 2017, sigue siendo un punto débil crítico. Este módulo, que se integra con SAP a través de APIs obsoletas (versión 2.1), solo permite la sincronización de datos una vez al día mediante procesos batch. La infraestructura actual incluye servidores con Windows Server 2012 R2 para módulos legacy no migrados.

Un ejemplo concreto de cómo esto afecta las operaciones diarias en Panamá se presenta cuando un pedido urgente de un cliente importante ingresa al sistema a las 10:00 AM. Debido a la sincronización batch, el pedido no se refleja en el sistema ERP hasta las 2:00 AM del día siguiente, lo que resulta en un retraso de 16 horas en la preparación y despacho del pedido. Esto provoca pérdidas significativas en ventas y afecta negativamente la satisfacción del cliente. Alicorp ha registrado un incremento del 15\% en los reclamos por entregas tardías en Panamá, en comparación con otros mercados donde la infraestructura está mejor adaptada.

La infraestructura de red en Panamá también presenta serios desafíos. Alicorp utiliza una red VPN privada basada en OpenVPN para conectar las instalaciones logísticas en Panamá con los servidores centrales en Sudamérica. Esta red se basa en routers Cisco 2901, que no han sido actualizados desde 2015, y opera con un ancho de banda limitado de 20 Mbps. La latencia promedio en condiciones normales es de 250 ms, pero puede aumentar hasta 500 ms durante picos de uso, lo que dificulta la sincronización en tiempo real de datos críticos como inventario y órdenes de compra.

Durante las horas pico, especialmente entre las 9:00 AM y 12:00 PM, la red en Panamá experimenta caídas frecuentes, con al menos dos interrupciones por semana que duran entre 10 y 30 minutos cada una. Estas interrupciones afectan la capacidad de actualizar y visualizar datos en tiempo real, lo que genera errores en los pedidos y un incremento del 12\% en la cantidad de órdenes incorrectas o retrasadas. Además, la inestabilidad de la red ha causado un aumento del 10\% en el tiempo de resolución de problemas logísticos, afectando la eficiencia global de las operaciones en el país.

\subsection{Definición del problema}

La expansión de Alicorp hacia El Salvador enfrenta desafíos significativos en el área de logística debido a limitaciones tanto en la arquitectura de TI como en la infraestructura de red. La obsolescencia del módulo SCM y la sincronización batch de datos causan retrasos considerables en la gestión de pedidos, afectando negativamente la eficiencia operativa y la satisfacción del cliente. La infraestructura de red, caracterizada por una VPN privada con routers desactualizados y ancho de banda limitado, presenta problemas de latencia y caídas frecuentes, que exacerbán los problemas en la sincronización de datos y la resolución de problemas logísticos. Para garantizar una expansión exitosa y una operación eficiente en El Salvador, Alicorp debe abordar estos desafíos mediante la actualización de su infraestructura de TI y la mejora de su red, permitiendo así una integración en tiempo real y una gestión logística más efectiva.