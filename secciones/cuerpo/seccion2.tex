\section{Estructura Organizacional en el área de TI}

%%%%%%%%%%%%%%%%%%%%%%%%%%%%%%%%%%%%%%%%%%%%%%%%%%%%%%%%%%%%%%%%%%%%%%%%%%%%%%%%%
%	                   Políticas de TI 
%%%%%%%%%%%%%%%%%%%%%%%%%%%%%%%%%%%%%%%%%%%%%%%%%%%%%%%%%%%%%%%%%%%%%%%%%%%%%%%%%

\subsection{Políticas de TI}
\lipsum[1]
%%%%%%%%%%%%%%%%%%%%%%%%%%%%%%%%%%%%%%%%%%%%%%%%%%%%%%%%%%%%%%%%%%%%%%%%%%%%%%%%%
%	                   Objetivos de TI 
%%%%%%%%%%%%%%%%%%%%%%%%%%%%%%%%%%%%%%%%%%%%%%%%%%%%%%%%%%%%%%%%%%%%%%%%%%%%%%%%%

\subsection{Objetivos de TI}
\lipsum[1]
\subsection{Organización del área de TI}
\lipsum[1]

%%%%%%%%%%%%%%%%%%%%%%%%%%%%%%%%%%%%%%%%%%%%%%%%%%%%%%%%%%%%%%%%%%%%%%%%%%%%%%%%%
%	                   Descripción de las áreas de TI
%%%%%%%%%%%%%%%%%%%%%%%%%%%%%%%%%%%%%%%%%%%%%%%%%%%%%%%%%%%%%%%%%%%%%%%%%%%%%%%%%

\subsection{Descripción de las áreas de TI}
\subsubsection{Área de TI}
\subsubsection{Área de Infraestructura de TI}
\lipsum[1]
\subsubsection{Área de proyectos de TI}
\lipsum[1]
\subsubsection{Área de mantenimiento: }

%%%%%%%%%%%%%%%%%%%%%%%%%%%%%%%%%%%%%%%%%%%%%%%%%%%%%%%%%%%%%%%%%%%%%%%%%%%%%%%%%
%	                   CIO 
%%%%%%%%%%%%%%%%%%%%%%%%%%%%%%%%%%%%%%%%%%%%%%%%%%%%%%%%%%%%%%%%%%%%%%%%%%%%%%%%%

\subsection{CIO}
\subsubsection{Funciones}
    \begin{itemize}
        \item Gestionar el personal del área de informática. 
        \item Negociar las relaciones con los proveedores. 
        \item Supervisar la arquitectura de TI. 
        \item Definir las políticas, normas y procesos de gobernanza de las TI. 
        \item Gestionar el riesgo de la información. 
        \item Tomar decisiones sobre el gasto e inversión en TI. 
        \item Gestionar la capacidad y el ciclo de vida de la tecnología. 
        \item Comprender las tendencias tecnológicas y su aplicabilidad a los objetivos de la empresa. 
        \item Comunicar la gestión de incidentes importantes a los ejecutivos y otras partes interesadas. 
        \item Encargarse del cumplimiento de normativas gubernamentales en materia de TI 
    \end{itemize}
\subsubsection{Responsabilidades }
    \begin{itemize}
        \item Impulsar la innovación tecnológica dentro del sector en el que se desenvuelve la compañía. 
        \item Asegurar el funcionamiento continuo y el rendimiento óptimo de las plataformas y sistemas esenciales para la operación del negocio. 
        \item Trabajar de manera conjunta con los equipos de desarrollo y operaciones para garantizar la implementación puntual de soluciones tecnológicas. 
        \item Supervisar la formación y el crecimiento profesional del equipo de tecnología. 
        \item Identificar y mitigar los riesgos asociados con la seguridad de la información y los sistemas tecnológicos. 
        \item Forjar relaciones estratégicas con proveedores clave de tecnología y servicios. 
        \item Contribuir en la planificación financiera y el control de gastos en el área tecnológica. 
        \item Comunicar de forma clara y efectiva la estrategia tecnológica y los avances logrados a las partes interesadas tanto internas como externas. 
        \item Analizar cómo la tecnología influye en la productividad y eficiencia de la empresa, sugiriendo mejoras de manera constante. 
    \end{itemize}
\subsubsection{Perfil de un CIO}
El perfil de un CIO (Chief Information Officer) en una organización de gran tamaño debe reunir una sólida formación en tecnologías de la información, una visión estratégica clara y habilidades de liderazgo excepcionales. Este ejecutivo debe contar con una experiencia comprobada en la implementación de soluciones tecnológicas innovadoras que optimicen los procesos operativos y faciliten la transformación digital de la empresa. Además, es esencial que tenga la capacidad de liderar equipos multidisciplinarios, manejar proyectos tecnológicos complejos y comunicarse eficazmente con las diferentes partes involucradas. 
El CIO debe estar preparado para identificar y adoptar nuevas tecnologías de manera proactiva, que impulsen la eficiencia y competitividad de la compañía, garantizando al mismo tiempo la seguridad de la información y el cumplimiento de las normativas vigentes. También es crucial que mantenga una mentalidad estratégica que asegure la alineación de la tecnología con los objetivos corporativos, así como la creación de alianzas estratégicas con proveedores y socios clave. En resumen, el CIO ideal debe ser un líder visionario, con una sólida formación técnica y una clara orientación hacia la innovación y la eficiencia en la implementación de soluciones tecnológicas. 
\paragraph*{Especificaciones del Perfil }
    \begin{itemize}
        \item Experiencia previa en la implementación de soluciones tecnológicas en la industria de consumo masivo, debe haber liderado proyectos exitosos en la implementación de tecnología para optimizar la cadena de suministro, la producción y la distribución de productos en empresas de gran escala. Es esencial tener un historial comprobado en la mejora de procesos logísticos, automatización de plantas y sistemas de distribución. 
        \item Conocimiento profundo de las tecnologías de la información aplicadas a la producción y logística, debe estar familiarizado con tecnologías como IoT, automatización industrial, inteligencia artificial y análisis de datos, plataformas de gestión de inventario y logística, así como sistemas ERP y CRM utilizados en la industria de consumo masivo. 
        \item Habilidades de liderazgo comprobadas en entornos complejos y de alto rendimiento, debe ser capaz de liderar equipos multidisciplinarios que incluyan profesionales de TI, ingeniería, y operaciones, creando un ambiente colaborativo y orientado a resultados dentro de una industria que opera a gran escala. 
        \item Experiencia en la gestión de la seguridad de la información y protección de datos en un entorno de operaciones complejas, asegurar el cumplimiento de las normativas de seguridad de la información y la protección de datos, en especial en la privacidad del cliente y la integridad de los sistemas de producción y distribución. 
        \item Capacidad para desarrollar e implementar estrategias tecnológicas alineadas con los objetivos de crecimiento de la empresa, debe ser capaz de impulsar la innovación tecnológica que mejore la eficiencia de la producción y la distribución, optimizando procesos operativos y mejorando la agilidad en la respuesta a las demandas del mercado. 
        \item Excelentes habilidades de comunicación, es crucial que pueda colaborar eficazmente con otros líderes empresariales dentro de la organización, como las áreas de operaciones, finanzas y marketing, así como interactuar con socios estratégicos y proveedores clave en el sector de tecnología. 
        \item Experiencia en la gestión de proyectos tecnológicos complejos dentro del sector de alimentos y productos de consumo masivo: El CIO debe asegurar que los proyectos tecnológicos, como la modernización de plantas de producción o la implementación de sistemas logísticos, se entreguen a tiempo y dentro del presupuesto. 
        \item Visión estratégica para identificar oportunidades de crecimiento y optimización tecnológica: Debe poder proponer y liderar iniciativas tecnológicas que maximicen la eficiencia y el crecimiento del negocio, centrando en la sostenibilidad y la innovación en la producción y distribución de productos. 
        \item Entendimiento profundo de los desafíos del sector de consumo masivo: El CIO debe estar familiarizado con las particularidades del mercado, las tendencias de consumo y los desafíos que enfrenta una empresa de la escala de Alicorp, siendo capaz de ofrecer soluciones tecnológicas que potencien su competitividad y liderazgo en el mercado. 
    \end{itemize}
%%%%%%%%%%%%%%%%%%%%%%%%%%%%%%%%%%%%%%%%%%%%%%%%%%%%%%%%%%%%%%%%%%%%%%%%%%%%%%%%%
%	                  Diagnóstico de la Situación Actual de las TICS 
%%%%%%%%%%%%%%%%%%%%%%%%%%%%%%%%%%%%%%%%%%%%%%%%%%%%%%%%%%%%%%%%%%%%%%%%%%%%%%%%%


\subsection{Diagnóstico de la Situación Actual de las TICS}
% \subsubsection{Capacidades de las TIC’s }
% \lipsum[1]
% \subsubsection{Valoración de las capacidades de las TIC’s }
% \lipsum[1]
\subsubsection{Requerimientos del negocio}
\lipsum[1]
% \subsubsection{Requerimientos de las TIC’s}
% \lipsum[1]
% \subsubsection{Proveedores }
% \lipsum[1]
