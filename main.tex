\documentclass{article}[12pt]
\usepackage{indentfirst}
% \usepackage{fontspec}
\usepackage{helvet}
% \usepackage[spanish,es-tabla, es-ucroman]{babel}

\usepackage{calc}
\usepackage{float}
\usepackage{caption}
\usepackage{graphicx}
\usepackage{subfigure}
\usepackage{mathtools}
\usepackage{amsmath}
\usepackage{pgfplotstable} 
\usepackage{natbib}
\usepackage{url}
\usepackage{enumerate} 
\usepackage{amsmath}
\usepackage{graphicx}
\graphicspath{{figuras/}}
\usepackage{parskip}
\usepackage{fancyhdr}
\usepackage{changepage}
\usepackage{xcolor}
\usepackage{tikz}
\usepackage{multicol}
\usepackage[most]{tcolorbox}
\usepackage{lipsum}

\usepackage{hyperref}

\usepackage{siunitx}
\usepackage{hyperref}

\usepackage{setspace}

\usepackage[absolute,overlay]{textpos}

\usepackage[a4paper, total={6in, 8in},textheight=240mm]{geometry}


\renewcommand{\rmdefault}{phv}


% CONFIGURACIÓN DE LA HOJA
% \setmarginsrb{4 cm}{2.5 cm}{2 cm}{2.5 cm}{1 cm}{1.5 cm}{1 cm}{0.5 cm}

% CONFIGURACIÓN DE SANGRÍA
% \setlength{\parindent}{4mm}
% \setlength{\parskip}{1mm}

\onehalfspacing % Espaciado de 1.5 líneas

%---------------------------------------------------------------------------
%	EN TODAS HOJAS
%---------------------------------------------------------------------------

\title{Alicop}				    	% Titulo
\author{Diseño e Implementación de Arquitectura Empresarial}	            % Nombre del estudiante				
\date{\today}						        % Fecha


\makeatletter
\let\thetitle\@title
\let\theauthor\@author
\let\thedate\@date
\makeatother

\pagestyle{fancy}
\lhead{
    \begin{tikzpicture}[overlay, fill]
        \fill[ColorFondoPrimario](-4,0) rectangle(8.5,1.5);    
        \fill[ColorFondoSecundario](9,0) rectangle(10,1.5);    
    \end{tikzpicture} 
}
\rhead{
    \begin{tikzpicture}[overlay, fill]
        \node[] at (-0.2, 0.4) 
    {\includegraphics[width=0.8cm,height=1cm]{marca_agua.png}};
    \end{tikzpicture} 
}

\rfoot{
    \begin{tikzpicture}[overlay, fill]
        \fill[ColorFondoSecundario](-20,-0.5) rectangle(5,-2.5);    
    \end{tikzpicture} 
}
\cfoot{\thepage}

\begin{document}


%%%%%%%%%%%%%%%%%%%%%%%%%%%%%%%%%%%%%%%%%%%%%%%%%%%%%%%%%%%%%%%%%%%%%%%%%%%%%%%%%
%	                                CARÁTULA 
%%%%%%%%%%%%%%%%%%%%%%%%%%%%%%%%%%%%%%%%%%%%%%%%%%%%%%%%%%%%%%%%%%%%%%%%%%%%%%%%%

\pgfplotsset{compat=1.16}
% CARÁTULA
\definecolor{ColorPrimario}{RGB}{233  103  103}
\definecolor{ColorSecundario}{RGB}{120 185 40}
\definecolor{ColorFondoPrimario}{RGB}{233  103  103}
\definecolor{ColorFondoSecundario}{RGB}{145  214  92}
\begin{titlepage}
  \thispagestyle{empty}
  \newgeometry{left=2cm, right=1cm, top=2cm, bottom=2cm}

  \begin{tikzpicture}[overlay,fill]
    \fill[ColorFondoPrimario](-5,10) rectangle(20,-35);    
  \end{tikzpicture}
  
  \begin{tikzpicture}[remember picture,overlay]
    % draw image
    \node[draw] at (11.5, -10) 
    {\includegraphics[width=16cm,height=22cm]{./figuras/background_alicorp.jpg}};
  \end{tikzpicture}
  
  \begin{tikzpicture}[overlay, fill,opacity=0.95]
    \fill[ColorFondoPrimario](-2.5,-28) rectangle(8.5,5);    
    \fill[white](-2.5, 1.5) rectangle(20,2);    
    \fill[white](-2.5,-17.5) rectangle(20,-18);    
    \fill[ColorFondoSecundario](-2.5,-28) rectangle(20,-18);    
  \end{tikzpicture}  


  \vspace{3 cm}

  \begin{minipage}{11cm}   

    \begin{flushleft}                               %alineado a ezquierda
    
    \Large \bfseries \textcolor{white}{DISEÑO E\\[0.1 cm] IMPLEMENTACIÓN DE LA }\\
    \Huge \bfseries \textcolor{white}{ARQUITECTURA EMPRESARIAL}\\
    
    \end{flushleft}

  \end{minipage}
  \vspace{\fill}

  %NOMBRE DE AUTOR
  \begin{minipage}{13cm}
    
    \begin{spacing}{1}
      \LARGE
      \begin{itemize}
        \item {Amaut Guzmán, Marcelo Fabian }
        \item {Baldeon Peña, Juancarlos }
        \item {Hernandez Ormeño, Roberth Alessandre }
        \item {Livise Larico, Rafael Enrique }
      \end{itemize}
    \end{spacing}
  \end{minipage}
  \begin{minipage}{4.5cm}
    \LARGE
    \begin{spacing}{1}
      \vspace{0.8cm}
      {U21230026}\\
      {U21201859}\\
      {1625004}\\
      {U17305723}\\
    \end{spacing}
  \end{minipage}

\end{titlepage}

%%%%%%%%%%%%%%%%%%%%%%%%%%%%%%%%%%%%%%%%%%%%%%%%%%%%%%%%%%%%%%%%%%%%%%%%%%%%%%%%%
%	                     INICIO DEL DOCUMENTO  
%%%%%%%%%%%%%%%%%%%%%%%%%%%%%%%%%%%%%%%%%%%%%%%%%%%%%%%%%%%%%%%%%%%%%%%%%%%%%%%%%



%%%%%%% ÍNDICE GENERAL
\tableofcontents
\thispagestyle{empty} % quita numero de la pagina

%%%%%%% INDICE DE FIGURAS
\newpage
\listoffigures
\thispagestyle{empty} % quita numero de la pagina

\newpage
\pagenumbering{arabic} % cera numero de la pagina

%%%%%%%%%%%%%% 
% SECCIÓN 1
%%%%%%%%%%%%%%
\section{Informcion de la empresa}

\subsection{Nombre de la empresa y logo}
ALICORP SAA\\ \\
\begin{figure}[!ht]
    \centering
    \includegraphics[scale = 0.6]{./figuras/logo_alicorp.png}	 
    \caption{Logo de Alicorp S.A.A}
\label{fig:logo}
\end{figure}


\subsection{Rubro o giro del negocio}

\subsection{Mision}
Transformamos mercados a través de nuestras marcas líderes, generando experiencias extraordinarias en nuestros consumidores. Buscamos innovar constantemente para generar valor y bienestar en la sociedad.

\subsection{Vision}
Ser líderes en los mercados en los que competimos.
\subsection{Productos o servicios}
La sociedad tiene por objeto social dedicarse a la industria, exportación, importación, distribución y comercialización de productos de consumo masivo

\begin{itemize}
\item Consumo masivo: Alimentos, cuidado del hogar y cuidado personal
\\Aceites y Grasas: Primor, Cocinero. 
\\Harinas y Pastas: Don Vittorio, Lavaggi. 
\\Productos de Cuidado del Hogar: Bolivar, Opal. 
\\Cuidado Personal: Plusbelle, Nutribelle, etc. 
\\Alimentos y Bebidas: Aunt Jemima, Fruttis, etc. 
\\Snacks y Dulces: Chizitos, Negrita, etc. 

\begin{figure}[!ht]
    \centering
    \includegraphics[width=0.8\textwidth]{./figuras/productos_marcas.jpg}
    \caption{Captura de la pagian web Alicop soluciones}
\label{fig:logo}
\end{figure}


\item Alicorp Soluciones: Ingredientes e insumos para los sectores de Panificación, Gastronomía y Grandes Industrias\\

\begin{figure}[!ht]
    \centering
    \includegraphics[width=0.8\textwidth]{./figuras/productos_alicorp_soluciones.png}
    \caption{Captura de la pagian web Alicop soluciones}
\label{fig:logo}
\end{figure}

\begin{figure}[!ht]
    \centering
    \includegraphics[width=0.8\textwidth]{./figuras/producto_insuma_app.png}
    \caption{Captura de la pagian web Alicop soluciones}
    \label{fig:logo}
\end{figure}

\item Acuicultura: Alimentos balanceados para camarón, salmón y peces. \\
Desde hace más de 30 años, Vitapro desarrolla soluciones especializadas en nutrición acuícola a través de sus marcas Nicovita y Salmofood, cumpliendo con los más altos estándares de calidad e innovando constantemente con el propósito de transformar la acuicultura para nutrir el mañana.
\\
\begin{figure}[!ht]
\centering
\includegraphics[width=0.6\textwidth]{./figuras/producto_nicovita.png}
\caption{Procuto nicovita}
\label{fig:logo}
\end{figure}

\end{itemize}


%%%%%%%%%%%%%% 
% SECCIÓN 2
%%%%%%%%%%%%%%

\section{Estructura Organizacional en el área de TI}

%%%%%%%%%%%%%%%%%%%%%%%%%%%%%%%%%%%%%%%%%%%%%%%%%%%%%%%%%%%%%%%%%%%%%%%%%%%%%%%%%
%	                   Políticas de TI 
%%%%%%%%%%%%%%%%%%%%%%%%%%%%%%%%%%%%%%%%%%%%%%%%%%%%%%%%%%%%%%%%%%%%%%%%%%%%%%%%%

\subsection{Políticas de TI}
\lipsum[1]
%%%%%%%%%%%%%%%%%%%%%%%%%%%%%%%%%%%%%%%%%%%%%%%%%%%%%%%%%%%%%%%%%%%%%%%%%%%%%%%%%
%	                   Objetivos de TI 
%%%%%%%%%%%%%%%%%%%%%%%%%%%%%%%%%%%%%%%%%%%%%%%%%%%%%%%%%%%%%%%%%%%%%%%%%%%%%%%%%

\subsection{Objetivos de TI}
\lipsum[1]
\subsection{Organización del área de TI}
\lipsum[1]

%%%%%%%%%%%%%%%%%%%%%%%%%%%%%%%%%%%%%%%%%%%%%%%%%%%%%%%%%%%%%%%%%%%%%%%%%%%%%%%%%
%	                   Descripción de las áreas de TI
%%%%%%%%%%%%%%%%%%%%%%%%%%%%%%%%%%%%%%%%%%%%%%%%%%%%%%%%%%%%%%%%%%%%%%%%%%%%%%%%%

\subsection{Descripción de las áreas de TI}
\subsubsection{Área de TI}
\subsubsection{Área de Infraestructura de TI}
\lipsum[1]
\subsubsection{Área de proyectos de TI}
\lipsum[1]
\subsubsection{Área de mantenimiento: }

%%%%%%%%%%%%%%%%%%%%%%%%%%%%%%%%%%%%%%%%%%%%%%%%%%%%%%%%%%%%%%%%%%%%%%%%%%%%%%%%%
%	                   CIO 
%%%%%%%%%%%%%%%%%%%%%%%%%%%%%%%%%%%%%%%%%%%%%%%%%%%%%%%%%%%%%%%%%%%%%%%%%%%%%%%%%

\subsection{CIO}
\subsubsection{Funciones}
    \begin{itemize}
        \item Gestionar el personal del área de informática. 
        \item Negociar las relaciones con los proveedores. 
        \item Supervisar la arquitectura de TI. 
        \item Definir las políticas, normas y procesos de gobernanza de las TI. 
        \item Gestionar el riesgo de la información. 
        \item Tomar decisiones sobre el gasto e inversión en TI. 
        \item Gestionar la capacidad y el ciclo de vida de la tecnología. 
        \item Comprender las tendencias tecnológicas y su aplicabilidad a los objetivos de la empresa. 
        \item Comunicar la gestión de incidentes importantes a los ejecutivos y otras partes interesadas. 
        \item Encargarse del cumplimiento de normativas gubernamentales en materia de TI 
    \end{itemize}
\subsubsection{Responsabilidades }
    \begin{itemize}
        \item Impulsar la innovación tecnológica dentro del sector en el que se desenvuelve la compañía. 
        \item Asegurar el funcionamiento continuo y el rendimiento óptimo de las plataformas y sistemas esenciales para la operación del negocio. 
        \item Trabajar de manera conjunta con los equipos de desarrollo y operaciones para garantizar la implementación puntual de soluciones tecnológicas. 
        \item Supervisar la formación y el crecimiento profesional del equipo de tecnología. 
        \item Identificar y mitigar los riesgos asociados con la seguridad de la información y los sistemas tecnológicos. 
        \item Forjar relaciones estratégicas con proveedores clave de tecnología y servicios. 
        \item Contribuir en la planificación financiera y el control de gastos en el área tecnológica. 
        \item Comunicar de forma clara y efectiva la estrategia tecnológica y los avances logrados a las partes interesadas tanto internas como externas. 
        \item Analizar cómo la tecnología influye en la productividad y eficiencia de la empresa, sugiriendo mejoras de manera constante. 
    \end{itemize}
\subsubsection{Perfil de un CIO}
El perfil de un CIO (Chief Information Officer) en una organización de gran tamaño debe reunir una sólida formación en tecnologías de la información, una visión estratégica clara y habilidades de liderazgo excepcionales. Este ejecutivo debe contar con una experiencia comprobada en la implementación de soluciones tecnológicas innovadoras que optimicen los procesos operativos y faciliten la transformación digital de la empresa. Además, es esencial que tenga la capacidad de liderar equipos multidisciplinarios, manejar proyectos tecnológicos complejos y comunicarse eficazmente con las diferentes partes involucradas. 
El CIO debe estar preparado para identificar y adoptar nuevas tecnologías de manera proactiva, que impulsen la eficiencia y competitividad de la compañía, garantizando al mismo tiempo la seguridad de la información y el cumplimiento de las normativas vigentes. También es crucial que mantenga una mentalidad estratégica que asegure la alineación de la tecnología con los objetivos corporativos, así como la creación de alianzas estratégicas con proveedores y socios clave. En resumen, el CIO ideal debe ser un líder visionario, con una sólida formación técnica y una clara orientación hacia la innovación y la eficiencia en la implementación de soluciones tecnológicas. 
\paragraph*{Especificaciones del Perfil }
    \begin{itemize}
        \item Experiencia previa en la implementación de soluciones tecnológicas en la industria de consumo masivo, debe haber liderado proyectos exitosos en la implementación de tecnología para optimizar la cadena de suministro, la producción y la distribución de productos en empresas de gran escala. Es esencial tener un historial comprobado en la mejora de procesos logísticos, automatización de plantas y sistemas de distribución. 
        \item Conocimiento profundo de las tecnologías de la información aplicadas a la producción y logística, debe estar familiarizado con tecnologías como IoT, automatización industrial, inteligencia artificial y análisis de datos, plataformas de gestión de inventario y logística, así como sistemas ERP y CRM utilizados en la industria de consumo masivo. 
        \item Habilidades de liderazgo comprobadas en entornos complejos y de alto rendimiento, debe ser capaz de liderar equipos multidisciplinarios que incluyan profesionales de TI, ingeniería, y operaciones, creando un ambiente colaborativo y orientado a resultados dentro de una industria que opera a gran escala. 
        \item Experiencia en la gestión de la seguridad de la información y protección de datos en un entorno de operaciones complejas, asegurar el cumplimiento de las normativas de seguridad de la información y la protección de datos, en especial en la privacidad del cliente y la integridad de los sistemas de producción y distribución. 
        \item Capacidad para desarrollar e implementar estrategias tecnológicas alineadas con los objetivos de crecimiento de la empresa, debe ser capaz de impulsar la innovación tecnológica que mejore la eficiencia de la producción y la distribución, optimizando procesos operativos y mejorando la agilidad en la respuesta a las demandas del mercado. 
        \item Excelentes habilidades de comunicación, es crucial que pueda colaborar eficazmente con otros líderes empresariales dentro de la organización, como las áreas de operaciones, finanzas y marketing, así como interactuar con socios estratégicos y proveedores clave en el sector de tecnología. 
        \item Experiencia en la gestión de proyectos tecnológicos complejos dentro del sector de alimentos y productos de consumo masivo: El CIO debe asegurar que los proyectos tecnológicos, como la modernización de plantas de producción o la implementación de sistemas logísticos, se entreguen a tiempo y dentro del presupuesto. 
        \item Visión estratégica para identificar oportunidades de crecimiento y optimización tecnológica: Debe poder proponer y liderar iniciativas tecnológicas que maximicen la eficiencia y el crecimiento del negocio, centrando en la sostenibilidad y la innovación en la producción y distribución de productos. 
        \item Entendimiento profundo de los desafíos del sector de consumo masivo: El CIO debe estar familiarizado con las particularidades del mercado, las tendencias de consumo y los desafíos que enfrenta una empresa de la escala de Alicorp, siendo capaz de ofrecer soluciones tecnológicas que potencien su competitividad y liderazgo en el mercado. 
    \end{itemize}
%%%%%%%%%%%%%%%%%%%%%%%%%%%%%%%%%%%%%%%%%%%%%%%%%%%%%%%%%%%%%%%%%%%%%%%%%%%%%%%%%
%	                  Diagnóstico de la Situación Actual de las TICS 
%%%%%%%%%%%%%%%%%%%%%%%%%%%%%%%%%%%%%%%%%%%%%%%%%%%%%%%%%%%%%%%%%%%%%%%%%%%%%%%%%


\subsection{Diagnóstico de la Situación Actual de las TICS}
% \subsubsection{Capacidades de las TIC’s }
% \lipsum[1]
% \subsubsection{Valoración de las capacidades de las TIC’s }
% \lipsum[1]
\subsubsection{Requerimientos del negocio}
\lipsum[1]
% \subsubsection{Requerimientos de las TIC’s}
% \lipsum[1]
% \subsubsection{Proveedores }
% \lipsum[1]



%%%%%%%%%%%%%% 
% SECCIÓN 3
%%%%%%%%%%%%%%
\section{Planteamiento del problema}
\subsection{Antecedentes del problema}

Alicorp, una empresa líder en la producción y comercialización de productos de consumo masivo en América Latina, ha decidido expandir sus operaciones hacia El Salvador como parte de su estrategia para aumentar su presencia en mercados clave de Centroamérica. Esta expansión enfrenta desafíos significativos relacionados con la arquitectura de TI y redes, que actualmente requieren ajustes para adaptarse a las demandas del nuevo mercado.

La migración a SAP S/4HANA ha sido un paso crucial para mejorar la eficiencia y agilidad en los procesos de Alicorp. SAP HANA, como base de datos en memoria, permite velocidades de procesamiento más rápidas en comparación con bases de datos tradicionales. Sin embargo, la integración con el módulo de gestión de la cadena de suministro (SCM) enfrenta limitaciones debido al uso de APIs que podrían estar obsoletas, lo que puede ralentizar la sincronización de datos. Aunque la sincronización de datos mediante procesos batch es adecuada para muchos escenarios, la frecuencia de actualización puede no ser suficiente para ciertos procesos críticos.

La empresa de logística adquirida en El Salvador opera con una versión anterior de SAP ERP (ECC 6.0), lo que puede generar desafíos en la integración con SAP S/4HANA. Esta versión antigua del sistema presenta ciertas dificultades para la transferencia eficiente de datos, aunque las soluciones de integración y migración pueden mitigar estos problemas. La infraestructura TI de esta empresa utiliza servidores con Windows Server 2012 R2, que puede limitar la flexibilidad y el rendimiento necesarios para operaciones logísticas optimizadas.

Un escenario típico podría implicar que un pedido registrado en SAP S/4HANA se procese con un cierto retraso debido a la sincronización con el SCM, aunque las demoras podrían no ser extremas. La empresa de logística en El Salvador enfrenta dificultades en la actualización de su sistema SAP ERP, lo que puede causar algunas inconsistencias en la planificación de rutas y la gestión de pedidos. Esto se traduce en un incremento moderado en la cantidad de solicitudes de actualización de estado de pedidos con inconsistencias y un aumento en las quejas de clientes por entregas retrasadas.

La infraestructura de red en El Salvador también presenta desafíos. Alicorp utiliza una red VPN privada basada en OpenVPN para conectar las instalaciones logísticas en El Salvador con los servidores centrales en Sudamérica. Esta red utiliza routers Cisco 2901 y tiene un ancho de banda de 20 Mbps. Aunque la latencia promedio es de 250 ms, puede aumentar durante picos de uso, afectando la sincronización de datos en tiempo real. Durante las horas pico, la red en El Salvador puede experimentar algunas interrupciones, lo que afecta la capacidad de actualizar y visualizar datos en tiempo real, incrementando moderadamente los errores en pedidos y tiempos de resolución de problemas.

En el día a día, Alicorp enfrenta problemas específicos debido a la infraestructura de red en El Salvador. Por ejemplo, durante una jornada laboral normal, un pedido urgente registrado en el sistema a las 10:00 AM podría no ser procesado correctamente hasta las 2:00 AM del día siguiente debido a la alta latencia y a la sincronización ineficiente de datos. Durante las horas pico, como entre las 9:00 AM y 11:00 AM, la red puede experimentar interrupciones y un aumento en la latencia a 500 ms, lo que impide la actualización en tiempo real del estado de los pedidos. Esto lleva a errores en el registro de pedidos, con un aumento en las solicitudes de actualización con inconsistencias y retrasos en la resolución de problemas logísticos, afectando la capacidad de cumplir con las expectativas de los clientes y deteriorando el servicio general.

\subsection{Definición del problema}

La expansión de Alicorp hacia El Salvador enfrenta desafíos en logística debido a limitaciones en la arquitectura de TI y en la infraestructura de red. La obsolescencia parcial del módulo SCM y la sincronización de datos en batch pueden causar ciertos retrasos en la gestión de pedidos, afectando la eficiencia operativa y la satisfacción del cliente. La infraestructura de red, con routers relativamente antiguos y ancho de banda limitado, presenta problemas de latencia y algunas interrupciones, que agravan los desafíos en la sincronización de datos y la resolución de problemas logísticos. Para garantizar una expansión exitosa y una operación eficiente en El Salvador, Alicorp debe abordar estos desafíos mediante la actualización de su infraestructura de TI y la mejora de su red, permitiendo una integración más fluida y una gestión logística más efectiva.
%%%%%%%%%%%%%% 
% SECCIÓN 4
%%%%%%%%%%%%%%
\section{PLANTEAMIENTO DEL PROBLEMA }
\subsection{Antecedentes del problema}

En el contexto de la expansión de Alicorp hacia Centroamérica, específicamente en El Salvador, la empresa enfrenta desafíos significativos en la optimización de su arquitectura de aplicación y TI, especialmente en el área de logística. La plataforma móvil de Insuma, que facilita la gestión de logística y ventas B2B, debe integrar eficientemente con el sistema SAP para proporcionar información actualizada sobre inventarios y pedidos. Sin embargo, la alta latencia en la comunicación entre SAP y el frontend móvil, combinada con las limitaciones en la infraestructura de red, afecta la actualización y visualización de datos críticos. Estos problemas se agravan durante los picos de demanda, impactando negativamente la eficiencia operativa y la experiencia del usuario en el área logística.

En relación con la arquitectura de aplicación de la plataforma móvil de Insuma para el área de logística de Alicorp en El Salvador, se ha implementado una solución basada en tecnologías modernas para conectar el frontend móvil con el backend que interactúa con el sistema SAP. Los módulos de SAP necesarios incluyen SAP MM (Material Management) para la gestión de inventarios, SAP SD (Sales and Distribution) para la gestión de ventas y pedidos, y SAP WM (Warehouse Management) para la administración de almacenes. La arquitectura de software del frontend móvil se basa en React Native, que permite el desarrollo de una aplicación móvil nativa con una sola base de código. El backend, construido con Node.js y Express, actúa como intermediario entre el frontend y SAP, utilizando SAP Gateway (versión 7.50) para facilitar la integración con SAP HANA como base de datos.


Para la arquitectura de datos, se ha centrado en la implementación de SAP HANA como la base de datos principal, conocida por su capacidad para manejar grandes volúmenes de datos con alta velocidad de respuesta gracias a su almacenamiento en memoria. SAP HANA permite un procesamiento de datos en tiempo real y análisis de grandes conjuntos de datos con latencias muy bajas, lo cual es crucial para las operaciones logísticas de Alicorp. Para optimizar aún más la comunicación y reducir la latencia entre el backend y SAP, se ha introducido una capa de caching con Redis. Redis, que maneja datos en memoria y ofrece tiempos de respuesta extremadamente rápidos, almacena información frecuentemente solicitada, como inventarios y precios de productos, para evitar consultas repetidas a SAP HANA. Con Redis, la capacidad de gestionar datos en caché permite una recuperación rápida de información, mientras que SAP HANA gestiona grandes volúmenes de datos con una alta eficiencia. La aplicación móvil utiliza Axios, una biblioteca de JavaScript para solicitudes HTTP, para interactuar con el backend, accediendo a los datos almacenados en Redis para respuestas rápidas o directamente desde SAP HANA cuando se necesita la información más actualizada. 

Para la arquitectura de TI, el área de logística también emplea Google Cloud Platform (GCP) para mejorar la escalabilidad y la disponibilidad de sus servicios. Utiliza Google Compute Engine para la ejecución de instancias virtuales, específicamente máquinas virtuales N2 estándar con Ubuntu 20.04 LTS como sistema operativo. Estas instancias albergan tanto el backend de Node.js como la capa de caching de Redis. Google Cloud Storage se utiliza para el almacenamiento de datos no estructurados y archivos estáticos, mientras que Google Cloud SQL ofrece una base de datos relacional adicional con MySQL 8.0 para manejar datos complementarios. La arquitectura de GCP se basa en una distribución en múltiples zonas de disponibilidad (zonas de us-central1, us-east1 y us-west1) para asegurar la alta disponibilidad y la tolerancia a fallos.

El area de logística comparte una red VPN privada basada en OpenVPN para conectar sus instalaciones logísticas en El Salvador con los servidores centrales en Sudamérica. Esta red emplea routers Cisco 2901 y tiene un ancho de banda de 20 Mbps. Aunque la latencia promedio es de 250 ms, puede aumentar durante picos de uso, afectando la sincronización de datos en tiempo real. Durante las horas pico, la red en El Salvador puede experimentar algunas interrupciones, lo que afecta la capacidad de actualizar y visualizar datos en tiempo real, incrementando moderadamente los errores en pedidos y tiempos de resolución de problemas.

En el día a día de la operación de logística en Alicorp, los usuarios de la plataforma móvil de Insuma a menudo enfrentan problemas relacionados con la alta latencia en la comunicación entre el backend y SAP. Durante los picos de demanda, como en campañas de promociones especiales o eventos de ventas, la latencia puede aumentar considerablemente, lo que resulta en tiempos de respuesta más largos al acceder a información crítica, como la disponibilidad de productos o el estado de los pedidos. Por ejemplo, si un usuario intenta consultar el inventario en tiempo real para procesar un pedido urgente y la latencia es alta, puede experimentar retrasos significativos, mostrando datos desactualizados o incorrectos sobre la disponibilidad de productos. Esto no solo afecta la eficiencia operativa, sino que también impacta negativamente en la experiencia del cliente, que puede enfrentar largos tiempos de espera o errores en sus pedidos. Además, la carga de productos en la plataforma móvil, que depende de las actualizaciones provenientes de SAP, puede verse afectada por la sincronización lenta. Esto genera problemas adicionales, como la sobrecarga en el middleware o en los servidores de caching, exacerbando la latencia y retrasando la disponibilidad de los datos en la aplicación móvil, lo cual contribuye a una experiencia de usuario frustrante y a una menor eficiencia en el procesamiento de pedidos y gestión de inventarios.

\subsection{Definición del problema}
El área de logística de Alicorp enfrenta desafíos en la gestión de inventarios y pedidos debido a limitaciones en su arquitectura de aplicaciones y en su infraestructura de TI, afectando la integración de sistemas y la sincronización de datos. Estas deficiencias impactan negativamente los pilares de aplicación y TI, comprometiendo la eficiencia operativa.

%%%%%%%%%%%%%% 
% SECCIÓN 5
%%%%%%%%%%%%%%
\input{secciones/cuerpo/seccion5.tex}

%%%%%%%%%%%%%% 
% SECCIÓN 6
%%%%%%%%%%%%%%

% %%%%%%% 1 SUBSECCIÓN %%%%%%
% \subsection{Sobre las figuras}
% Las figuras deben cumplir con los siguientes criterios: \\

% \begin{itemize}
% \item Estar numeradas y con título en la parte inferior del texto.
% \item Deben ser explicadas y citadas con su respectivo numero antes de su presentación en el documento. 
% \item Tener buena resolución. En el caso en que hayan texto, este debe estar en el idioma local.
% \item Ser citadas con su respectivo número antes de su presentación en el documento. 
% \item Las gráficas deben tener los ejes con título, unidades de medida y leyendas si necesario. El tamaño del título de los ejes y de la leyenda deben ser visibles al lector.\\
% \end{itemize} 

% La Figura \ref{fig:logo} muestra un ejemplo.

% \begin{figure}[!ht]
% \centering
% \includegraphics[width=0.2\textwidth]{figuras/Isotipo_NEGRO.png}
% \caption{Logo UTEC}
% \label{fig:logo}
% \end{figure}

% %%%%%%% 2 SUBSECCIÓN %%%%%%

% \subsection{Sobre las tablas}
% Sigue algunos criterios a tener en cuenta en el momento de presentar tablas:\\

% \begin{itemize}
% \item Deben estar numeradas y con título en la parte inferior del texto.
% \item Deben ser explicadas y citadas con su respectivo número antes de su presentación en el documento. 
% \item Las magnitudes de estudio deben estar identificadas (ej. datos, sitio, potencia, y otros), acompañada de su unidad de medida siempre que haya. En caso de que se utilicen símbolos, estos deben ser explicados en el texto. 
% \item Los valores  deben ser escritos teniendo en cuenta las cifras significativas que indique el docente responsable.
% \item Para números muy grandes o pequeños se debe usar notación científica.\\
% \end{itemize}

% Ejemplos:

% \begin{table}[htb]
% \begin{center}
% \renewcommand\arraystretch{1.3}
% \centering
% \begin{tabular} {c c c c}
%  \hline
% N° & t (s) & I (W/m$^{2}$) & d (m)\\
%  \hline
% 1 & 1,23 & 4200 & 1,20\\
% 2 & 1,25 & 3997 & 1,50\\
% 3 & 1,23 & 4200 & 1,20\\
%  \hline
%  \end{tabular}
%  \caption{Valores experimentales de la práctica X}
% \label{ta:tab1}
%     \end{center}
%  \end{table}
 
% \begin{table}[htb]
% \begin{center}
% \begin{tabular}{l c c}
% \hline
% Mes & $\mathrm{HSP_{PH} (h)}$ & $\mathrm{HSP_{PI} (hrs)}$ \\
% \hline
% Enero & 224 & 213\\ 
% Julio & 224 & 213\\ 
% Diciembre & 224 & 213\\ 
% \hline
% \end{tabular}
% \caption{Ejemplo 2 de tabla}
%     \end{center}
% \label{tab:ValoresHSP}
% \end{table}

% %%%%%%%%%%%%%% 
% % SECCIÓN 7
% %%%%%%%%%%%%%%
% \section{Conclusiones}
% Este sección debe cerrar la propuesta del trabajo. Algunos criterios para la redacción: \\

% \begin{itemize}
% \item Ser objetivo. 
% \item Retomar la propuesta inicial y vincular con los resultados obtenidos.
% \item Realizar un análisis critico, proponer mejoras y estudios futuros.
% \end{itemize}

% %%%%%%%%%%%%%% 
% % SECCIÓN 8
% %%%%%%%%%%%%%%
% \section{Características de la bibliografía}
% La sección bibliográfica debe estar ordenada alfabéticamente por apellido del autor. Se recomienda la utilización del formato APA (American Psychological Association).

% %%%%%%%%%%%%%% 
% % SECCIÓN 9
% %%%%%%%%%%%%%%
% \section{Características de los anexos}

% \begin{itemize}
% \item Pueden ser documentos de distinta índole, que aporten información complementaria o aclaratoria de cierto aspecto de la obra. 
% \item Si fueron extraídos de fuentes consultadas, se deberán incluir sus referencias. 
% \item Deben estar incorporados en el orden en que fueron mencionados en el desarrollo del trabajo.
% \end{itemize}


\end{document}
